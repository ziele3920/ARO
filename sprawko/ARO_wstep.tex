\section{Wstęp}
	\subsection{Cel projektu}
	

	Celem projektu jest zaprojektowanie i przeprowadzenie badań jakości heurystycznych algorytmów rozpoznawania obrazów z uczeniem typu $NM$  (najbliższa średnia) oraz $\alpha-NN$ ($\alpha$-najbliższych sąsiadów, $ \alpha=1,3,4$) Na przykładzie zadania rozpoznawania obrazów jednowymiarowych ($d=1$) pochodzących z $M = 2$ klas, które występują z prawdopodobieństwem $p_1$ oraz $p_2 (p_1,p_2>0, p_1+p_2=1)$ a rozkład cech x w klasach jest odpowiednio:
	\begin{enumerate}
	\item równomierny $f_1(x)=U(a_1,b_1), f_2(x)=U(a_2,b_2)$
	\item normalny $f_1(x)=N(\mu_1, \sigma_1), f_2=N(\mu_2, \sigma_2)$
	\end{enumerate}
	
	\subsection{Opis badań}
	